\documentclass[a4paper,12pt]{article}

\usepackage[slovene]{babel}
\usepackage{amsfonts,amssymb,amsmath}
\usepackage[utf8]{inputenc}
\usepackage[T1]{fontenc}
\usepackage{lmodern}
\usepackage{graphicx}
\usepackage[usenames, dvipsnames]{color}


\def\qed{$\hfill\Box$}   % konec dokaza
\def\qedm{\qquad\Box}   % konec dokaza v matematičnem načinu
\newtheorem{izrek}{Izrek}
\newtheorem{trditev}{Trditev}
\newtheorem{posledica}{Posledica}
\newtheorem{lema}{Lema}
\newtheorem{opomba}{Opomba}
\newtheorem{definicija}{Definicija}
\newtheorem{zgled}{Zgled}
\newtheorem{dokaz}{Dokaz}
\newtheorem{algoritem}{Algoritem}

\title{\Huge Praštevila in njihove lastnosti \\ 
\Large Seminar}
\author{Sara Bizjak \\
Fakulteta za matematiko in fiziko \\
Oddelek za matematiko}
\date{April 2018}

\begin{document}
	
\maketitle

\newpage

\tableofcontents

\newpage

\section{Uvod}
S praštevili se vsi seznanimo že v zgodnjih letih šolanja, kar priča o tem, da
je njihova definicija enostavna in bi morala biti vsakomur razumljiva. Ravno
zaradi te preprostosti pa je presenetljivo, koliko vprašanj v zvezi s praštevili je
še vedno odprtih. Že samo ugotoviti praštevilnost je lahko precej zahtevno, saj
pri številih z nekaj tisoč števkami tudi Eratostenovo rešeto, kot verjetno najenostavnejši
postopek, postane prezamudno. Prav zaradi dejstva, da je za velika
števila težko preveriti, ali so praštevila ali ne, je ta tema tako zanimiva.
\\
\\
Tudi v tem članku se bomo pobliže seznanili z »večjimi« praštevili, dokazali
bomo obstoj neskončnega števila praštevil, nekaj izrekov in kako se praštevila
pojavljajo v množici naravnih števil.

\newpage

\section{Množica praštevil}
Praštevilo je število, ki ima natanko dva delitelja. Z izjemo števila $2$, ki je najmanjše
praštevilo, so vsa praštevila liha. $1$ ni praštevilo.

\begin{definicija}
	Praštevilo je naravno število $n > 1$, če število $n$ delita le števili $1$ in $n$.
\end{definicija}
Za občutek naštejmo prvih nekaj praštevil:
\\
\\
$\mathbb{P} = \{2, 3, 5, 7, 11, 13, 17, 19, 23, 29, 31, 37, 41, 43, 47, 53, 59, 61, 67, 71, 73, 79, 83, 89, 97, ... \} $
\\
\\
Vsako naravno število je »sestavljeno« iz praštevil, kar nam pove tudi naslednji
izrek, ki ga bomo dokazali kasneje.
\begin{izrek}
	(Osnovni izrek aritmetike) Vsako naravno število lahko na natanko
	en način zapišemo kot produkt samih praštevil. Takemu zapisu pravimo tudi razcep števila na prafaktorje in je enoličen do vrstnega reda natančno.
\end{izrek}

\subsection{Največji skupni delitelj}
Za dokaz osnovnega izreka aritmetike bomo potrebovali še nekaj pojmov in
premislekov, zato jih posebej navedimo. Ključni korak k dokazu bo razmislek,
da če praštevilo deli zmnožek dveh števil, potem deli ali prvo ali drugo število.
Za to pa potrebujemo vpeljavo pojma največjega skupnega delitelja.

\begin{definicija}
	Naj bo $\gcd(a,b)$ največji skupni delitelj števil $a$ in $b$, torej
	$$\emph{gcd}(a,b) = \emph{max}\{d \in \mathbb{N} : d|a \ in \ d|b\},$$
razen če sta $a$ in $b$ oba $0$--tedaj velja $\emph{gcd}(0,0)=0$.
\end{definicija}

\begin{zgled}
	Izračunani največji skupni delitelji:
	\begin{itemize}
		\item $\emph{gcd}(1,2)=1,$
		\item $\emph{gcd}(5,15)=5,$
		\item $\emph{gcd}(6,27)=3,$
		\item $\emph{gcd}(0,a)=\emph{gcd}(a,0)=a$ za vsak $a$.
	\end{itemize}
\end{zgled}

\begin{lema}
	Za vsaka $a,b \in \mathbb{Z}$ velja
	$$ \emph{gcd}(a,b) = \emph{gcd}(b,a) = \emph{gcd}(\pm a, \pm b) = \emph{gcd}(a, b - a) = \emph{gcd}(a, b + a).$$
\end{lema}

\noindent {\em Dokaz:\/}
	Dokažimo samo enakost gcd$(a, b)$ = gcd$(a, b-a)$.
	\\
	Naj bo $d$ skupni delitelj $a$ in $b$, torej $d|a$ in $d|b$, torej obstajata taki celi števili
	$c_1$ in $c_2$, da velja $dc_1 = a$ in $dc_2 = b$. Tedaj $b-a = dc_2-dc_1 = d(c_2-c_1)$ in
	takoj sledi, da $d|b-a$. Tako velja gcd$(a, b)$ $\le$ gcd$(a, b-a)$, saj je množica, iz
	katere jemljemo gcd$(a, b)$, podmnožica množice za gcd$(a, b-a)$.
	\\
	Če a zamenjamo z $-a$ in $b$ zamenjamo z $b-a$, z istim argumentom kot prej,
	dobimo: gcd$(a, b-a)$ = gcd$(-a, b-a)$ $\le$ gcd$(-a, b)$ = gcd$(a, b)$, iz česar sledi
	gcd$(a, b)$ = gcd$(a, b-a)$.
\qed

\begin{zgled}
	Privzemimo, da smo osnovni izrek aritmetike že dokazali.
	\\
	S pomočjo faktorizacije zračunajmo $\emph{gcd}(2261, 1275)$. $2261$ lahko zapišemo kot
	$7\cdot17\cdot19$ in $1275$ kot $3\cdot5\cdot5\cdot17$.
	\\
	Največji skupni delitelj je enak zmnožku skupnih prafaktorjev, 
	\\
	torej $\emph{gcd}(2261, 1275) = 17$.
\end{zgled}
Motivacija: Naj bosta $a, b \in \mathbb{Z}$ in $b \ne 0$. Tedaj obstajata natanko določena
$q, r \in \mathbb{Z}$, tako da $0 \le r < |b|$ in $a = bq + r$. \\
To nas pripelje do naslednjega algoritma,
imenovanega po starogrškem matematiku Evklidu. Prednost Evklidovega
algoritma je, da števil ni potrebno razcepiti. Sam postopek je eden najstarejših
znanih algoritmov in je znan od približno leta 300 pr. n. št., verjetno pa je bil
poznan že 200 let prej, in sicer kot algoritem za določanje največje skupne mere
dveh daljic.

\begin{algoritem}
	Naj bosta $a, b \in \mathbb{Z}$ in $b \ne 0$. Ta algoritem izračuna
	$q, r \in \mathbb{Z}$, tako da $0 \le r < |b|$ in $a = bq + r$. 
\end{algoritem}

\begin{opomba}
	Ker smo s tem algoritmom v večini seznanjeni že od prej, ne bomo
	opisali prav vseh korakov, ampak si bomo ogledali zgled dejanske uporabe.
\end{opomba}

\begin{zgled}
	Oglejmo si kar prejšnji primer, za katerega rešitev poznamo že iz klasičnega
	faktoriziranja. Izračunajmo torej $\emph{gcd}(2261, 1275)$ s pomočjo Evklidovega
	algoritma. Za lažje nadaljevanje označimo $a = 2261$ in $b = 1275$.
	\\
	Vidimo, da
	$$2261 = 1\cdot1275 + 986,$$
	torej je $q = 1$ in $r = 986$. Za nadaljevanje posodobimo spremenljivke--$a$ postane
	število $1275$, $b$ postane število $986$ ($b \longmapsto a \ in \ r \longmapsto b$) in dobimo:
	$$1275 = 1\cdot 986 + 289.$$
	Z istim postopkom nadaljujemo, dokler $r = 0$.
	$$986 = 3\cdot 289 + 119$$
	$$189 = 2\cdot 119 + 51$$
	$$119 = 2\cdot 51 + \textcolor{red}{17}$$
	$$51 = 3\cdot 17 + 0$$
	Ko je $r = 0$, zaključimo in preberemo največji skupni delitelj, ki ga najdemo
	korak oz. vrstico višje, na mestu $r$ (pobarvan rdeče).
	\\
	Tudi z Evklidovim algoritmom pridemo do istega rezultata kot prej, in sicer
	$\emph{gcd}(2261, 1275) = 17.$
\end{zgled}
Naslednjo lemo navedimo brez dokaza.

\begin{lema}
	Za $a,b,n \in \mathbb{Z}$ velja:
	$$\emph{gcd}(an,bn) = \emph{gcd}(a,b)\cdot|n|.$$
\end{lema}

\begin{lema}
	Naj bodo $a,b,n \in \mathbb{Z}$ in naj $n|a$ in $n|b$. Tedaj $n|\emph{gcd}(a,b).$
\end{lema}

\noindent{\em Dokaz:\/}
	Ker $n|a$ in $n|b$, obstajata $c$ in $d$, tako da $a=nc$ in $b=nd$. Po Lemi 2, gcd$(a, b)$ = gcd$(nc, nd)$ = $n \cdot$ gcd$(c, d)$, torej $n$ deli gcd$(a, b)$.
\qed
\\
\\
Če praštevilo deli produkt dveh števil, potem deli eno izmed števil. To lahko pokažemo z Evklidovim algoritmom in bo kasneje ključ pri dokazovanju osnovnega
izreka aritmetike.
	
\begin{izrek}
	Naj bo $p \in \mathbb{P}$ in $a,b \in \mathbb{N}$. Če $p|ab$, potem $p|a$ ali $p|b$.
\end{izrek}

\noindent {\em Dokaz:\/}
	Če $p|a$, smo končali. Če $p\nmid a$, potem gcd$(p, a)=1$. Po Lemi 2,
	gcd$(pb, ab) = b$. Ker $p|pb$ in po hipotezi $p|ab$, potem (po Lemi 2) velja:
	$p|$gcd$(pb, ab) = b\cdot$ gcd$(p, a) = b \cdot 1 = b$.
\qed

\subsection{Dokaz osnovnega izreka aritmetike}

V prvem delu dokaza bomo pokazali, da lahko vsako število zapišemo kot produkt
samih praštevil. Enoličnost faktorizacije (do vrstnega reda natančno), in s tem tudi dokončno osnovni
izrek aritmetike, pa bomo dokazali v drugem delu.
\\
\\
\noindent {\em Dokaz:\/}
\\
1. del:
\\
Naj bo $n \in \mathbb{N}$. Če $n = 1$, potem je $n$ prazen produkt praštevil. Če je $n$ praštevilo, smo že končali. Sicer $n$ zapišemo kot $ab$ in $a, b < n$. Po
indukciji sta tudi $a$ in $b$ produkta praštevil, torej je take oblike tudi $n$.
\\	
\\
2. del:
\\
Denimo, da imamo za število $n$ dve faktorizaciji. Privzamemo, da je $n > 1$.
Tedaj obstajajo taka praštevila $p_1, p_2, \ldots , p_m$, da $n = p_1 \cdot \ldots \cdot p_m$. Naj bo $n = q_1 \cdot \ldots \cdot q_n$ drugi zapis števila $n$ kot produkt praštevil. Po Evklidu je lahko $p_1 = q_1$ ali $p_1|q_2\cdots q_n$. Po indukciji vidimo, da velja $p_1 = q_i$ za nek $i$. Tako $p_1$ in $q_i$ ’okrajšamo’ in ponovimo postopek. Vidimo, da sta ti dve faktorizaciji enaki.
Enoličnost faktorizacije do vrstnega reda natančno je dokazana.
\qed

\section{Zaporedje praštevil}
Sedaj bomo odgovorili na sledeča vprašanja:
\begin{itemize}
	\item Koliko je vseh praštevil?
	\item Če imamo celi števili $a$ in $b$ podani, koliko je praštevil oblike $ax + b$?
	\item Kako so praštevila razporejena v množici naravnih števil?
\end{itemize}
Najprej bomo pokazali, da je praštevil neskončno, potem Dirichletov izrek, nazadnje
pa se bomo seznanili s funkcijo $\pi(x)$, ki nam pove, koliko praštevil, manjših od števila $x$, obstaja.

\subsection{Obstaja neskončno praštevil}

Motivacija: Poglejmo si spodnjo tabelo. Vsako število na levi strani enačaja je praštevilo. Sicer prav takšen postopek ne deluje v nedogled, deluje pa nekaj zelo podobnega.

\begin{align*}
3 &= 2 + 1 \\
7 &= 2 \cdot 3 + 1 \\
31 &= 2\cdot 3\cdot 5+1 \\
211 &= 2\cdot 3\cdot 5\cdot 7+1 \\
2311 &= 2\cdot 3\cdot 5\cdot 7\cdot 11+1 \\
\ldots
\end{align*}
Evklid je zapisal: Praštevil je več kot v kateremkoli izbranem seznamu praštevil.
Danes bi to lahko povedali na sledeč način.

\begin{trditev}
	Obstaja neskončno mnogo praštevil.
\end{trditev}

\noindent{\em Dokaz:\/}
	Naj bodo $p_1, p_2, \ldots, p_n$ različna praštevila. Naj bo $P = p_1p_2\ldots p_n + 1$ in
	naj bo $p$ praštevilo, ki deli $P$. Potem $p$ ne sme biti eno izmed praštevil
	$p_1, p_2, \ldots, p_n$, ker bi sicer število $p$ delilo razliko $P-p_1p_2\ldots p_n = 1$, kar pa je
	nemogoče. Torej je $p$ še eno novo praštevilo in $p_1, p_2, \ldots, p_n$ niso vsa praštevila.
	Praštevil je torej neskončno.
\qed
\\
\\
Ali po "Evklidovo":

\noindent{\em Dokaz:\/}
	Naj bodo A, B, C izbrana praštevila. Pravim, da je praštevil več kot
	A, B, C. Vzemimo število, ki ga merijo A, B, C, naj bo to DE, in dodajmo
	enoto DF številu DE. Potem EF ali je praštevilo ali pa ni.
	\\
	A\_\_\_\_\_\_\_\_\_ B \_\_\_\_\_\_\_\_\_\_\_\_\_\_\_\_\_\_\_\_\_\_C\_\_\_\_\_\_\_\_\_
	\\
	G\_\_\_\_\_\_\_\_\_\_\_
	\_\_\_\_\_\_\_\_\_\_\_\_\_\_\_\_\_\_\_ E\_\_\_\_\_\_\_\_\_\_\_
	\\
	\_\_\_\_\_\_\_\_\_\_\_\_\_\_\_\_\_\_\_\_\_\_
	\_\_\_\_\_\_\_\_\_\_\_\_\_\_\_\_
	D\_\_\_\_\_
	\\
	\_\_\_\_\_\_\_\_F
	\\
	Najprej naj bo praštevilo, potem smo našli praštevila A, B, C, EF, ki jih je več
	kot A, B, C. Potem naj EF ne bo praštevilo, torej ga meri neko praštevilo.
	Naj ga meri praštevilo G. Pravim, da je G različen od vseh A, B, C.
	Predpostavimo nasprotno. Vemo, da A, B, C merijo DE. Torej tudi G meri
	DE. Meri pa tudi EF. Zato mora G, ki je število, deliti tudi preostalo enoto
	DF, kar je protislovje. Tako je G različen od vseh A, B, C. In po hipotezi je
	praštevilo. Zato smo našli števila A, B, C, G, ki jih je več od izbranih A, B, C.
\qed
\\
\\
Poznamo tudi druge dokaze, ki pokažejo isto stvar. Nekateri taki so Eulerjev,
Polyajev, Kummerjev dokaz, pa tudi topološki dokaz.
Od teh je zelo zanimiv Eulerjev dokaz, ki ga je predstavil leta 1737. To je bilo
namreč prvič, da sta aritmetika in analiza, do tedaj popolnoma ločena pomena
proučevanja, začeli delovati skupaj. Od takrat dalje sta teorija števil in analiza
neločljivi vedi.
Zapišimo še Eulerjev dokaz:
\\
\\
\noindent{\em Dokaz:\/}
	Euler je poznal vsoto geometrijskega zaporedja
	\\
	$1 + \alpha z + \alpha^2z^2 + \alpha ^3z^3 + \ldots = \frac{1}{1-\alpha z}$.
\\
	Ulomek $\frac{1}{(1-\alpha z)(1-\alpha z)}$ lahko torej zapišemo kot produkt dveh vsot
	\\
	$(1 +\alpha z +\alpha^2z^2 +\alpha^3z^3 + \ldots)$ in $(1 +\beta z +\beta^2z^2 +\beta^3z^3 +\ldots)$, zato je enak
	$1 + (\alpha+\beta)z + (\alpha^2+\alpha\beta+\beta^2)z^2 + (\alpha^3+\alpha^2\beta+\alpha\beta^2+\beta^3)z^3 + \ldots $
	\\
	Na ta način je razvil ulomek $\frac{1}{(1-\alpha z)(1-\beta z)(1-\gamma z)(1-\delta z)\ldots}$ v obliko
	\\
	$1+Az+Bz^2+Cz^3+Dz^4+ ...$, kjer je $A$ vsota koeficientov $\alpha, \beta, \gamma, \delta, \ldots$, B je
	vsota produktov po dveh koeficientih, C vsota produktov po treh koeficientih
	itd.
	\\
	Vzemimo z = 1, koeficienti $\alpha, \beta, \gamma, \ldots $ pa naj zavzamejo recipročne vrednosti
	praštevil $2, 3, 5, 7,\ldots$ Zahvaljujoč osnovnemu izreku aritmetike je Euler tako
	dobil neskončno harmonično vrsto $1 + \frac{1}{2} + \frac{1}{3} + \frac{1}{4} + \ldots$, za katero je trdil, da je
	enaka $\log(\frac{1}{1-x})$ pri $x = 1$.
	\\
	Prvotni produktni ulomek je tako neskončen, če sprejmemo Eulerjevo nekoliko
	drzno upravljanje z neskončnimi količinami brez upoštevanja limit, in je enak
	dolžini seznama praštevil.
\qed
\\
\\
Kljub temu, da je praštevil neskončno, se je smiselno vprašati, katero je največje
znano praštevilo. Mersennovo praštevilo je število oblike $2^q-1$. Največje najdeno praštevilo je 
$$ 2^{74207281}-1.$$
To število ima več kot 22 milijonov mest

\subsection{Praštevila oblike $ax+b$}

Oglejmo si praštevila oblike $ax + b$, kjer sta $a > 1$ in $b$ fiksni celi števili in $x$
teče po naravnih številih. Privzamemo, da gcd$(a, b) = 1$, saj sicer ne bi bilo
neskončno praštevil take oblike. Na primer, $2x + 2 = 2(x + 1)$ je praštevilo le,
če $x = 0$ in ne za vsak $x \in \mathbb{N}$.
\\
\\
Poglejmo si praštevila oblike $4x-1$. Zapišimo jih prvih nekaj in obarvajmo
tiste, ki so praštevila.
$$ \textcolor{red}{3}, \textcolor{red}{7}, \textcolor{red}{11}, 15, \textcolor{red}{19}, \textcolor{red}{23}, 27, \textcolor{red}{31}, 35, 39, \textcolor{red}{43}, \textcolor{red}{47},...$$

\begin{izrek}
	Obstaja neskončno praštevil oblike  $4x-1$.
\end{izrek}

\noindent{\em Dokaz:\/}
	Naj bodo $p_1, p_2, ..., p_n$ praštevila forme $4x-1$. Naj bo
	$P = 4p_1p_2...p_n-1$. Tedaj $p_i \nmid P$. Ni vsako število $p$, ki deli $P$, oblike $4x + 1$.
	Če bi bilo, bi bil tudi $P$ oblike $4x + 1$. Ker je $P$ liho število, je vsak
	praštevilski delitelj $p_i$ lih, torej obstaja $p|P$ oblike $4x-1$. Ta postopek lahko
	ponavljamo v nedogled, torej je množica praštevil oblike $4x-1$ neskončna.
\qed
	
\begin{zgled}
	Naj bo $p_1 = 3$ in $p_2 = 7$. Tedaj je
	$$P = 4 \cdot 3 \cdot 7-1 = \textcolor{red}{83}$$
	praštevilo forme $4x-1$. Če računamo naprej, je
	$$P = 4 \cdot 3 \cdot 7 \cdot 83-1 = \textcolor{red}{6971}$$
	praštevilo forme $4x-1$. V naslednjem koraku dobimo
	$$P = 4 \cdot 3 \cdot 7 \cdot 83 \cdot 6971-1 = 48601811 = 61 \cdot \textcolor{red}{796751},$$
	kjer je $796751 = 4 \cdot 199188-1$.
	\begin{center}
	...
	\end{center}
\end{zgled}	
Poglejmo si še Dirichletov izrek brez dokaza.
\begin{izrek}
	Naj bosta $a$ in $b$ taki celi števili, da $gcd(a, b) = 1$. Tedaj obstaja neskončno
	mnogo praštevil oblike $ax + b$.
\end{izrek}

\subsection{Gostota praštevil}

Videli smo že, da je praštevil neskončno mnogo, torej je vprašanje, koliko je vseh
praštevil, nesmiselno. Smiselno pa je vprašanje o gostoti oziroma o deležu praštevil
v naravnih številih. Kot je na primer polovica vseh naravnih števil ravno
sodih oziroma lihih, poskušamo podobno ugotoviti za praštevila. Vprašajmo se
raje, koliko je praštevil manjših ali enakih nekemu naravnemu številu.

\begin{definicija}
	Definirajmo
	$$\pi(x) = \#\{p \in \mathbb{N}: p \le x, \ kjer \ p \in \mathbb{P} \}.$$
	Funkcija $\pi(x)$ je torej moč množice z vsemi praštevili manjšimi ali enakimi $x$.
\end{definicija}

\begin{zgled}
	$\pi(6) = \#\{2,3,5\}=3$.
\end{zgled}

\noindent Poglejmo tabelo in graf z vrednostmi $\pi(x) \ za \ x < 1000$.

\begin{table}[ht]
	\centering
	\label{my-label}
	\begin{tabular}{|l|l|}
		\hline
		$x$  & $\pi(x)$ \\ \hline
		100  & 25       \\
		200  & 46       \\
		300  & 62       \\
		400  & 78       \\
		500  & 95       \\
		600  & 109      \\
		700  & 125      \\
		800  & 139      \\
		900  & 154      \\
		1000 & 168      \\ \hline
	\end{tabular}
	\caption{Vrednosti $\pi(x)$}
\end{table}

\begin{figure}[!ht]
	\centering
	\includegraphics[width=13cm]{oj.png}
	\caption{Graf funkcije $\pi(x)$ za $x < 1000$}
\end{figure}

\newpage
\noindent Motivacija: Kolikšna je verjetnost, da če naključno izberemo število med $0$
in nekim številom $x$, bo to izbrano število ravno praštevilo? Na to nam odgovori
praštevilski izrek, ki v grobem pravi, da je ta verjetnost enaka približno
$1/ \log(x)$. Torej praštevilski izrek govori o asimptotični porazdelitvi praštevil.
Podaja splošni opis, kako so praštevila porazdeljena med pozitivnimi celimi števili.
Povzame intuitivno zamisel, da se z večanjem števila $x$ praštevila pojavljajo
vse redkeje.

\begin{izrek}
	Funkcija $\pi(x)$ je asimptotična funkciji $\frac{x}{\log(x)}$, tako da
	$$\lim_{x\to\infty}\frac{\pi(x)}{\frac{x}{\log(x)}} = 1.$$
\end{izrek}
Vidimo, da Izrek 5 pravzaprav pove, da je $\pi(x)$ približno enako $x/ \log(x)$ v smislu,
da se relativna napaka tega približka približuje $0$, ko gre $x$ preko vsake
meje:
$$\lim_{x\to\infty}\frac{\pi(x)}{x} = \lim_{x\to\infty}\frac{1}{\log(x)} = 0,$$
torej za vsak a velja:
$$\lim_{x\to\infty}\frac{\pi(x)}{\frac{x}{\log(x)-a}} = \lim_{x\to\infty}\frac{\pi(x)}{\frac{x}{\log(x)}} - \frac{a\pi(x)}{x}= 1.$$
\\
Še več, tudi $\frac{x}{\log(x)-a}$ je asimptotična (približno enaka) $\pi(x)$ za vsak $a$. Če izberemo
$a = 1$, vidimo, da je to najboljša možnost (privzeto).
\\
\newpage
\noindent Poglejmo si še tabelo, ki primerja vrednosti $\pi(x)$ in $\frac{x}
{\log(x)-1}$ za nekaj vrednosti
$x < 10000$.
\begin{table}[!ht]
	\centering
	\label{my-label}
	\begin{tabular}{|l|l|l|}
		\hline
		$x$   & $\pi(x)$ & $\frac{x}{\log(x)-1}$ (aproksimacija) \\ \hline
		1000  & 168      & 169,2690290604408165786256278         \\
		2000  & 303      & 302,9888734545463878029800994         \\
		3000  & 430      & 428,1819317975237043747385740         \\
		4000  & 550      & 548,3922097278253264133400985         \\
		5000  & 669      & 665,1418784486502172369455815         \\
		6000  & 783      & 779,2698885854778626863677374         \\
		7000  & 900      & 891,3035657223339974352567759         \\
		8000  & 1007     & 1001,602962794770080754784281         \\
		9000  & 1117     & 1110,428422963188172310675011         \\
		10000 & 1229     & 1217,976301461550279200775705         \\ \hline
	\end{tabular}
	\caption{Primerjava $\pi(x)$ in $\frac{x}{\log(x)-1}$}
\end{table}


\noindent Kot zanimivost si poglejmo metodologijo dokaza praštevilskega izreka avstralsko-ameriškega
matematika Terenca Chi-Shen Taa.
V predavanju o praštevilih za širšo javnost je Tao, prejemnik Fieldsove medalje
leta 2006, opisal pesniški pristop dokaza praštevilskega izreka s poslušanjem
praštevilske glasbe. Začne se z zvočnim valovanjem, ki je glasno pri praštevilih
in tiho pri sestavljenih številih--to je von Mangoldtova aritmetična funkcija.
Potem se analizira njegove note oziroma frekvence s procesom, sorodnim Fourierjevi
transformaciji--to je Mellinova transformacija. Potem se dokaže, kar je
težek del, da se določene note v tej glasbi ne morejo pojaviti. Ta izključitev
določenih not vodi do praštevilskega izreka. Po Tau ta dokaz vodi do globljega
vpogleda v porazdelitev praštevil kot elementarni dokazi.

\newpage
\section{Zanimivosti}
\subsection{Največja znana praštevila nekoč in danes}
Že antični Grki so vedeli, da je praštevil neskončno mnogo, a prav velikih praštevil
niso poznali.
Prvi praštevili, ki ju že lahko uvrščamo med »večja« praštevila, sta $2^{17}-1 =
131071$ in $2^{19}-1 = 524287$, ki ju je leta 1588 pravilno preveril italijanski matematik
Pietro Cataldi. Cataldi je poznal praštevila med 2 in korenom zgornjih
dveh, torej je moral preveriti samo, da nista deljivi z nobenim manjšim praštevilom.
Domneval je, da so tudi števila oblike $2^n-1$ za $n = 23, 29, 31, 37$
praštevila. Prav je imel le za število $2^{31}-1$, kar je leta 1772 dokazal Euler.
Do prave revolucije pri iskanju velikih praštevil je prišlo leta 1876, ko je francoski
matematik Eduard Lucas odkril domiselen in preprost kriterij, ki je prejšnje
postopke precej olajšal. Dokazal je, da je 39-mestno število $2^{127}-1$ praštevilo.
Ta rezultat je že predstavljal uvod v računalniško dobo iskanja velikih praštevil.
Danes si iskanje velikih praštevil brez pomoči računalnika ne moremo niti predstavljati.
Po zaslugi Lucasove ugotovitve so računalniki še posebej uspešni pri
iskanju velikih Mersennovih praštevil (praštevila oblike $2^n-1$). Tako je med
največjimi danes poznanimi praštevili največ Mersennovih.

\subsection{Praštevilski dvojčki}
Poleg iskanja praštevil je zelo zanimivo in težko iskanje t. i. praštevilskih dvojčkov.
\\
Pravimo, da praštevili $p$ in $q$ tvorita praštevilski dvojček, če se po absolutni
vrednosti razlikujeta za natanko 2. Tako so praštevilski dvojčki na primer $(3,5),
\ (5,7), \ (11,13), \ (17,19)...$
\\
Razen pri paru $(2,3)$ je to najmanjša možna razlika
med dvema prašteviloma.
Vprašanje, ali obstaja neskončno mnogo praštevilskih dvojčkov, je že vrsto let
eno od velikih odprtih vprašanj v teoriji števil. Leta 2004 je Richard Arenstorf
z Vanderbiltove univerze v Nashvilleu, Tennessee v članku na 38 straneh podal
dokaz, da obstaja neskončno mnogo praštevilskih dvojčkov, vendar je mesec
dni kasneje Michel Balazard z Univerze v Bordeauxu pokazal na napako in
Arenstorf je moral umakniti dokaz.

\newpage
\section{Zaključek}
Ker je praštevil neskončno, smo priča vedno novemu lovu na največje znano praštevilo. To iskanje je predvsem zabava za ljudi, ki so jim všeč števila. Za dokazovanje pa je potrebno veliko več znanja, razvijati se morajo tudi tehnologija in algoritmi. Praštevila niso zanimiva le za nas matematike, ampak so uporabna predvsem v računalniških vedah. Zaradi odkrivanja vedno večjih praštevil so računalniški sistemi varnejši, kar temelji na kriptografiji. Kriptografija v osnovi poišče dve dokaj veliki praštevili, ki ju je preprosto zmnožiti. A ko imamo produkt, je težko ugotoviti, katera sta njegova praštevilska faktorja. Varnost temelji na tem, da kdor hoče vdreti v sistem, produkta ne zna razbiti na praštevili, učinkovitih algoritmov, ki pa bi nam na to dali odgovor, še ni.	
\\
Glede na uporabo praštevil v vsakdanjem življenju bo to v naslednjih letih še zelo zanimiva tematika.

\newpage	
\begin{thebibliography}{99}
	\bibitem{a} 
	William Stein: Elementary Number Theory: Primer, Congruences, and
	Secrets, dostopno na https://wstein.org/ent/ent.pdf.
	
	\bibitem{b} Wikipedia, praštevila.
	
	\bibitem{c} Revija Presek, list za mlade matematike, fizike, astronome in računalničarje,
	DMFA-založništvo, 6, 349-351, Letnik 28.
	
	\bibitem{d} http://www.educa.fmf.uni-lj.si/izodel/sola/2003/ura/valentincic/zaporedje.html.
	
	\bibitem{e} https://www.rtvslo.si/stevilke/statistika/najvecje-najdeno-prastevilo-ima-vec-kot-22-milijonov-mest/393403.
	
	
\end{thebibliography}
\end{document}	

